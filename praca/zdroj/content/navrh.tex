\chapter{Návrh aplikácie}

\section{Prípady použitia}

\section{Jazyk -- Ruby}

Ruby je dynamický, reflektívny objektovo orientovaný programovací jazyk, ktorý kombinuje syntax inšpirovanú jazykmi Perl, Smalltalk, Eiffel, Lisp a má dynamickú správu pamäti. Autor -- Yukihiro Matsumoto -- pri návrhu vychádzal z filozofie, že programátor by sa pri práci s jazykom mal baviť ale zároveň byť produktívny. Taktiež zdôrazňoval, že dizajn systémov by sa mal zameriavať viac na potreby človeka ako na potreby počítača. \citep{matsumoto2002ruby}

\subsection{Zaujímavé vlastnosti jazyka Ruby}

Ruby je veľmi zaujímavý jazyk, avšak má mnoho vlastností, ktoré by mohli samostatne pokryť ďalšiu prácu. Preto som sa rozhodol vybrať také, ktoré sú netradičné a priblížiť ich.

\subsubsection{Symbol}

Symbol na prvý pohľad vyzerá ako premenná, ktorá má pred názvom dvojbodku, napríklad \emph{:user}. Výhodou symbolov je, že ich nie je potrebné deklarovať a v celom programe budú mať rovnakú hodnotu. Zoberme nasledujúci príklad:

\begin{minted}{ruby}
puts "string".object_id
puts "string".object_id
puts :symbol.object_id
puts :symbol.object_id
\end{minted}

Po spustení tohto programu sa zobrazil nasledujúci výstup a naozaj dokázal, že symboly majú počas behu programu rovnakú hodnotu:

\begin{minted}{bash}
$ ruby symbol.rb
90520360
90520080
801628
801628
\end{minted}

\subsubsection{Blok}

Bloky sú v iných jazykoch nazývané aj \emph{closure}. V Ruby sú veľmi užitočnou vlastnosťou pretože nám umožňujú odoslať blok kódu priamo do funkcie. Definujú sa:

\begin{minted}{ruby}
call_block(arg) { |a| puts a }
# alebo
call_block(arg) do |a|
  puts a
end
\end{minted}

Pre využitie bloku vo funkcii musíme použiť funkciu \emph{yield}. Avšak, ak funkcia používa \emph{yield} ale, žiadny block nebol prijatý tak Ruby vyvolá výnimku. Blok môžeme vo funkcii zavolať nasledujúco:

\begin{minted}{ruby}
def call_block(arg)
    yield(arg)
end
\end{minted}

\subsubsection{Moduly}

Ruby nepodporuje viacnásobnú dedičnosť. Keď chceme aby niekoľko tried implementovalo určitú funkcionalitu musíme využiť \emph{module}. Module je kolekcia metód a konštánta môžeme modul definovať nasledujúco:

\begin{minted}{ruby}
module Mod
  def call_func(arg)
    puts arg
  end
end
\end{minted}

V triede kde požadujeme funkcionalitu modulu potom jednoducho zahrnieme modul pomocou funkcie \emph{include}. Napríklad:

\begin{minted}{ruby}
class MyClass
    include Mod
end
\end{minted}

\subsection{RubyGems}

RubyGems je správca balíčkov používaný spolu s jazykom Ruby, ktorý ponúka štandardný formát pre distribúciu Ruby programov a knižníc vo formáte nazývanom \emph{gem} a uľahčuje ich inštaláciu. Všetky balíčky sa dajú nájisť, prehľadávať alebo pridávať na stránke \url{https://rubygems.org}. V čase písania práce bolo publikovaných 131 tisíc balíčkov a počet stiahnutí presiahol 13,3 miliárd.

\section{Framework -- Ruby on Rails}

\subsection{Model, view a controller}
\subsection{ActiveRecord}
\subsection{Migrácie}
\subsection{Generátory}

\section{Databáza a entitno-relačný diagram}

Na perzistentné uloženie dát aplikácie som po porade zvolil databázu SQLite, ktorá má podľa mňa dostatočnú kapacitu a spĺňa všetky požiadavky pre škálu budovanej aplikácie. Má dokonca aj niekoľko výhod:

\begin{itemize}
    \item Jednoduchá inštalácia
    \item Nie je potrebná skoro žiadna konfigurácia
    \item Dáta sú ľahko zálohovateľné, pretože databáza je uložená v jedinom súbore
    \item Šetrí zdroje systému
\end{itemize}

\section{Systém pre správu verzií -- Git}

Pri tvorbe softvéru vývojári vytvárajú zdrojový kód. Menia, rozširujú ho, potrebujú vrátiť zmeny alebo sa vrátiť späť k starším verziám. Taktiež je potrebné zabezpečiť situáciu, keď na jednom projekte pracuje viac vývojárov naraz. \citep{otte2009version} 

Git je jedným z mnohých programov na správu verzií zdrojového kódu vytvorený Linusom Torvaldom v roku 2005. Používať ho budem počas celého vývoja aplikácie aj napriek tomu, že nie som súčasťou žiadneho týmu, pretože mi stále poskytuje niekoľko obrovských výhod, a to: 

\begin{itemize}
    \item Jednoduché a prehľadné zdieľanie zdrojového kódu s vedúcim práce
    \item História všetkých zmien zdrojového kódu
    \item Uľahčenie nasadenia na server
    \item Záloha aplikácie v prípade straty na lokálnom systéme
\end{itemize}