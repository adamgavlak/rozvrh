\chapter*{Úvod}
\addcontentsline{toc}{chapter}{Úvod}

%- informatizácia spoločnosti
%- využívanie aplikácii v akademickom prostredí
%- internet je jeden z hlavných zdrojov informácii
%- technológie na vývoj web apps pribúdajú stále
%- prečo vytváram aplikáciu
%- načo bude dobrá (katedra potrebuje aplikáciu k istým interným procesom - výpočet mzdy jednotlivých vyučujúcich)

%- popísanie obsahu práce
%- teoretická časť (analýza požiadaviek, návrh)
%- havlná časť - implementácia
%- práca bola niekoľko krát konzultovaná a robená na mieru práve pre KMMOA
%- čo píšem v závere

Na počiatku bol Internet používaný hlavne pre akademické účely, ale za roky prevádzky sa rozvinul do dnešnej gigantickej podoby, keď už každý z nás môže mať prístup k Internetu aj vo vrecku, pričom sa stále rozrastá každým dňom.

Podobnou rýchlosťou sa vyvíjajú aj nástroje na vývoj webových aplikácii a v posledných rokoch sa začínajú deliť na oddelené disciplíny. Jednou z mojich úloh bolo porovnať a zvoliť si technológiu, na ktorej postavím internú webovú aplikáciu pre Katedru matematických metód a operačnej analýzy.

Keďže celý vývoj musím zvládnuť sám rozhodol zvoliť z obrovského množstva dnes dostupných technológii jazyk Ruby a použiť framework Ruby on Rails, pretože už s ním mám skúsenosti, poskytne mi všetky potrebné nástroje na splnenie daných cieľov a verím že mi uľahčí efektívny vývoj webovej aplikácie, kde sa môžem lepšie sústrediť na spracovanie požiadaviek koncových používateľov.

V tejto práci analyzujem vytvorenie kompletnej webovej aplikácie pre podporu tvorby katedrových rozvrhov, od analýzy a návrhu až po samotné nasadenie aplikácie na produkčný server. Konečným cieľom tejto práce je webová aplikácia, ktorá dokáže vypočítať úväzky pre jednotlivých vyučujúcich katedry, generovať prehľady do PDF súborov a následne ich aj odoslať danému vyučujúcemu cez e-mail.