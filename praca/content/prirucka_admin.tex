\section*{Príručka administrátora}
\addcontentsline{toc}{section}{Príručka administrátora}

Pri nasadení bude nutné nainštalovať niekoľko nástrojov, pre správny chod aplikácie a jej závislostí. Príručka je napísaná pre distribúciu Debian 8 (Jessie), ale mala by fungovať pre väčšinu Unix-like operačných systémov.

Začneme nainštalovaním všetkých potrebných balíčkov:

\begin{minted}{bash}
$ apt update
$ apt install git-core curl zlib1g-dev build-essential libssl-dev 
  libreadline-dev libyaml-dev libsqlite3-dev sqlite3 libxml2-dev 
  libxslt1-dev libcurl4-openssl-dev python-software-properties 
  libffi-dev

\end{minted}

\clearpage
\subsection*{Inštalácia manažéra prostredia Ruby -- rbenv}

Rbenv je veľmi užitočný nástroj, ktorý umožní jednoducho inštalovať a meniť verziu Ruby. Nainštalovaný je len lokálne pre užívateľa, ktorý vykonal inštaláciu. Nainštalujeme ho nasledovne:

\begin{minted}{bash}
$ cd
$ git clone https://github.com/rbenv/rbenv.git ~/.rbenv
$ echo 'export PATH="$HOME/.rbenv/bin:$PATH"' >> ~/.bashrc
$ echo 'eval "$(rbenv init -)"' >> ~/.bashrc

$ git clone https://github.com/rbenv/ruby-build.git ~/.rbenv/plugins/ruby-build
$ echo 'export PATH="$HOME/.rbenv/plugins/ruby-build/bin:$PATH"' >> ~/.bashrc
$ source ~/.bashrc

$ rbenv install 2.4.0
$ rbenv global 2.4.0
\end{minted}

Teraz môžeme overiť, či sa Ruby správne nainštaloval pomocou príkazu

\begin{minted}{bash}
$ ruby -v
\end{minted}

Po úspešnom nainštalovaní rbenv a Ruby nainštalujeme balíček Bundler, ktorý nám bude manažovať závislosti v aplikáciach pomocou súboru \emph{Gemfile}.

\begin{minted}{bash}
$ gem install bundler
$ rbenv rehash
\end{minted}

\clearpage
\subsection*{Príprava aplikácie na serveri}

Teraz je čas stiahnuť našu aplikáciu na server (odporúčam naklonovať Git repozitár). Aplikáciu umiestnime jednoducho do \emph{/home/UZIVATEL/APLIKACIA}. Potom prejdeme do adresára kde sme aplikáciu umiestnili a spustíme príkaz:

\begin{minted}{bash}
$ bundle install
\end{minted}

Ďalší krok je vytvorenie databázy. SQLite vytvorí súbor databázy a následne tabuľky v adresári \emph{db} pomocou príkazov:

\begin{minted}{bash}
$ rake db:create
$ rake db:migrate
\end{minted}

Príkaz (\emph{rake db:create}) nám databázu vytvorí v zložke db v adresári aplikácie a \emph{rake db:migrate} vytvorí tabuľky podľa migrácii.

Posledná časť prípravy aplikácie je spustenie balíčka \emph{\texttt{delayed\_job}}, aby sa mohli spúšťať úlohy na pozadí. To spravíme spustením príkazu z adresára aplikácie:

\begin{minted}{bash}
$ bin/delayed_job start
\end{minted}

Po týchto krokoch je aplikácia pripravená a môžeme prejisť na inštaláciu webového servera.

\subsection*{Inštalácia webového servera Nginx}

Naša aplikácia bude využívať webový server Nginx s plugin-om \emph{Passenger} od spoločnosti \emph{Phusion}. V prvom kroku nainštalujeme (ako \emph{sudo}) PGP kľúč a pridáme HTTPS podporu pre APT repozitára:

\begin{minted}{bash}
$ apt-key adv --keyserver hkp://keyserver.ubuntu.com:80 
  --recv-keys 561F9B9CAC40B2F7
$ apt install apt-transport-https ca-certificates
\end{minted}

V ďalšom kroku pridáme samotný repozitár a aktualizujeme list repozitárov

\begin{minted}{bash}
$ echo 'deb https://oss-binaries.phusionpassenger.com/apt/passenger 
  jessie main' > /etc/apt/sources.list.d/passenger.list
$ apt update
\end{minted}

V poslednom kroku nainštalujeme samotný web server Nginx aj spolu s Passenger-om

\begin{minted}{bash}
$ apt install nginx-extras passenger
$ service nginx start
\end{minted}

Keď otvoríme IP stroja vo webovom prehliadači, mala by sa zobraziť štandardná stránka nainštalovaného serveru Nginx. \citep{web:phusion-passenger}


\subsection*{Konfigurácia webového servera Nginx}

Súbory možno editovať v akomkoľvek textovom editore. V príklade je použitý editor \emph{Joe}.

Nájdeme nasledujúci riadok v súbore \emph{/etc/nginx/nginx.conf} a odkomentujeme ho:

\begin{minted}{text}
include /etc/nginx/passenger.conf;
\end{minted}

Je potrebné zmeniť užívateľa pod ktorým bude Nginx bežať. V rovnakom súbore (úplne navrchu) zmeníme \emph{user www-data;} na užívateľa, ktorý ma nainštalované Ruby s pomocou rbenv nasledovne:

\begin{minted}{text}
user UZIVATEL;
\end{minted}

Po úprave súbor uložíme a zavrieme. Otvoríme súbor \emph{/etc/nginx/passenger.conf} a zmeníme riadok ktorý začína \emph{\texttt{passenger\_ruby}} na pričom zameníme UZIVATEL za užívateľa ktorý aplikáciu inštaluje na:

\begin{minted}{text}
passenger_ruby /home/UZIVATEL/.rbenv/shims/ruby;
\end{minted}

Nginx je teraz nastavený, aby pri Ruby aplikáciach využíval aktuálne nainštalovanú verziu Ruby s pomocou rbenv. Posledný krok je konfigurácia prepojenie Nginx-u na aplikáciu. Po otvorení \emph{/etc/nginx/sites-enabled/default} obsah zmeníme na:

\begin{minted}{text}
server {
        listen 80;

        server_name domena.sk;
        passenger_enabled on;
        rails_env    production;
        root         /home/UZIVATEL/APLIKACIA/public;

        # presmerovanie errorov na statickú stránku /50x.html
        error_page   500 502 503 504  /50x.html;
        location = /50x.html {
            root   html;
        }
}
\end{minted}

Samozrejme, je potrebné zameniť UZIVATEL za meno užívateľa, pod ktorým aplikácia beží a APLIKACIA za meno aplikácie, prípadne cestu kde je aplikácia uložená. Reštartujeme Nginx, aby prejavili zmeny v konfigurácii:

\begin{minted}{bash}
$ service nginx restart
\end{minted}

A teraz aplikácia beží na zadanej doméne, prípadne IP servera.