\chapter*{Záver}
\addcontentsline{toc}{chapter}{Záver}

Práca prináša krátky prehľad technológii na strane klienta a na strane servera a hlavne oboznámenie s programovacím jazykom Ruby a frameworkom Ruby on Rails. Popisuje ako postupovať pri vývoji webovej aplikácie s použitím tohto frameworku, bežnú štruktúru projektu a nástroje ktoré prichádzajú spolu s Ruby alebo Ruby on Rails. Dopĺňam aj ako som ja postupoval pri riešení konkrétnych problémov, na ktoré som pri vývoji narazil.

Aj pri vývoji jednoduchšieho projektu ako je tento, sa mi potvrdilo že Ruby a framework Ruby on Rails sú výbornými nástrojmi nielen na veľké projekty ale dajú sa využiť aj pri menšom internom projekte a aj na rýchle prototypovanie webových aplikácii. Keďže Ruby on Rails je založený na princípe -- konvencia pred konfiguráciou -- verím, že ak by niekto chcel moju aplikáciu rozšíriť alebo upraviť môže tak urobiť bez väčších problémov.

Výsledná webová aplikácia po niekoľkých stretnutiach s koncovými používateľmi a dlhom vývoji spĺňa požadovanú funkcionalitu, je stabilná a verím, že aspoň trochu odbremení zamestnancov katedry matematických metód a operačnej analýzy, umožní im sústrediť sa na dôležitejšie problémy alebo postupovať vo výskume a do istej miery zautomatizuje vytváranie, generovanie a zdieľanie prehľadov úväzkov pre vyučujúcich.